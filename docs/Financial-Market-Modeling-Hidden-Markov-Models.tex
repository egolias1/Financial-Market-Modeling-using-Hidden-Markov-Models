\documentclass[a4paper,11pt]{article}
\pdfoutput=1 % if your are submitting a pdflatex (i.e. if you have
             % images in pdf, png or jpg format)

\usepackage{jheppub} % for details on the use of the package, please
                     % see the JHEP-author-manual

\usepackage[T1]{fontenc} % if needed



\title{Financial Market Modeling using Hidden Markov Models}


%% %simple case: 2 authors, same institution
%% \author{A. Uthor}
%% \author{and A. Nother Author}
%% \affiliation{Institution,\\Address, Country}

% more complex case: 4 authors, 3 institutions, 2 footnotes
\author{Elliot Golias}
%\author[c]{S. Econd,}
%\author[a,2]{T. Hird\note{Also at Some University.}}
%\author[a,2]{and Fourth}

% The "\note" macro will give a warning: "Ignoring empty anchor..."
% you can safely ignore it.

%\affiliation[a]{One University,\\some-street, Country}
%\affiliation[b]{Another University,\\different-address, Country}
%\affiliation[c]{A School for Advanced Studies,\\some-location, Country}

% e-mail addresses: one for each author, in the same order as the authors
\emailAdd{elliot.golias@gmail.com}
%\emailAdd{second@asas.edu}
%\emailAdd{third@one.univ}
%\emailAdd{fourth@one.univ}




\abstract{This project applies stochastic models in quantitative research 
and analysis to accurately price options and assess associated risks. Leveraging historical financial data, 
it encompasses data preprocessing, model selection, parameter estimation, option pricing, risk analysis, 
and sensitivity analysis. Selected models like Black-Scholes-Merton and Heston are calibrated using 
optimization techniques. Accurate option prices and key option Greeks are calculated based on 
estimated parameters and underlying asset prices. Risk analysis utilizes Monte Carlo simulation 
to evaluate metrics such as VaR and ES. Sensitivity analysis explores the impact of changing model 
parameters. The project emphasizes documentation, resulting in a comprehensive report. It showcases 
expertise in option valuation, risk assessment, and quantitative analysis, demonstrating proficiency 
in data preprocessing, model implementation, and advanced quantitative techniques.}



\begin{document} 
\maketitle
\flushbottom

%%%%%%%%%%%%%%%%%%%%%%%%%%%%%%%%%%%%%%%%%%%%%%%
\section{Introduction}
%%%%%%%%%%%%%%%%%%%%%%%%%%%%%%%%%%%%%%%%%%%%%%%
\label{sec:intro}

%%%%%%%%%%%%%%%%%%%%%%%%%%%%%%%%%%%%%%%%%%%%%%%
\section{Data Collection}
%%%%%%%%%%%%%%%%%%%%%%%%%%%%%%%%%%%%%%%%%%%%%%%
\label{sec:data_collection}

%%%%%%%%%%%%%%%%%%%%%%%%%%%%%%%%%%%%%%%%%%%%%%%
\section{Stochastic Models: Black-Scholes-Merton and Heston Models}
%%%%%%%%%%%%%%%%%%%%%%%%%%%%%%%%%%%%%%%%%%%%%%%
\label{sec:models}

The Black-Scholes-Merton model results from a solution from the stochastic differential equation

\begin{align}
  \label{eq:BSM-SDE}
  \d S(t) = \mu S(t) \ d t + \sigma S(t) \ d W(t)
\end{align}

where $S(t)$ is the price of the asset at time $t$, $\mu$ is the drift or expected return of the asset per unit time,
$\sigma$ is the volatility of the asset per unit time, and $W(t)$ is a Wiener process.

It is important to note that the the Black-Scholes-Merton stochastic differential equation
makes several key assumtions:

\begin{itemize}
  \item Efficient Markets: Markets are efficient and there are no opportunities for arbitrage.
  \item Constant Parameters: The parameters, specifically the risk-free interest rate $r$, 
  the volatility of the asset $\sigma$, and the dividen yield $q$, of this model are constant with time
  \item Continuous Time: Asset prices and other factors change continuously rather than discretely.
  \item Log-Normal Distribution: The asset price follows a log-normal distribution. 
  \item No Transaction Costs: There are no brokerage fees, taxes, or other costs associated with the 
  underlying asset or option.
  \item No Dividends: The underlying asset does not pay any dividends during the option's lifetime.
        If the asset does pay dividends, they are assumed to be continuous and accounted for by adjusting the drift
        term in the stochastic differential equation describing the Black-Scholes model.
  \item No Market Frictions: There are no liquidity constraints, market impact, trading restrictions, or
  other market frictions.
  \item Risk-Neutral Pricing: The expected return for the asset is equal to the risk-free rate. This assumption
        allows for discounting future payoffs at the discounted rate.
\end{itemize}

For a European call option, the theoretical price $C$ is given by

\begin{align}
  C = S_t N(d_1) - X e^{-rT} N(d_2)
\end{align},

and the theoretical price $P$ of a European put option is given by

\begin{align}
  P = X e^{-r T} N(d_2) - S_t N(d_1)
\end{align}

where $S_t$ is the current price of the underlying asset, $X$ is the option's strike price,
$r$ is the risk-free interest rate, $T$ is the time to expiration in years,
$N(x)$ is the cumulative standard normal distribution function, and

\begin{align}
  d_1 &= \frac{\ln{\frac{S_t}{X}} + (r + \sigma^2/2)*T}{\sigma \sqrt{T}} \\
  d_2 &= d_1 - \sigma \sqrt{t}
\end{align}

where $\sigma$ is the volatility or standard deviation of the asset's returns.

The Heston model for option pricing extends the Black-Scholes-Merton model by incorporating
a dynamic process for volatility.


%%%%%%%%%%%%%%%%%%%%%%%%%%%%%%%%%%%%%%%%%%%%%%%
\section{Conclusions}
%%%%%%%%%%%%%%%%%%%%%%%%%%%%%%%%%%%%%%%%%%%%%%%
\label{sec:conclusions}


%\appendix
%\section{Some title}
%Please always give a title also for appendices.





%\acknowledgments

%This is the most common positions for acknowledgments. A macro is
%available to maintain the same layout and spelling of the heading.

%\paragraph{Note added.} This is also a good position for notes added
%after the paper has been written.





% The bibliography will probably be heavily edited during typesetting.
% We'll parse it and, using the arxiv number or the journal data, will
% query inspire, trying to verify the data (this will probalby spot
% eventual typos) and retrive the document DOI and eventual errata.
% We however suggest to always provide author, title and journal data:
% in short all the informations that clearly identify a document.

%\begin{thebibliography}{99}

%\bibitem{a}
%Author, \emph{Title}, \emph{J. Abbrev.} {\bf vol} (year) pg.

%\bibitem{b}
%Author, \emph{Title},
%arxiv:1234.5678.

%\bibitem{c}
%Author, \emph{Title},
%Publisher (year).


% Please avoid comments such as "For a review'', "For some examples",
% "and references therein" or move them in the text. In general,
% please leave only references in the bibliography and move all
% accessory text in footnotes.

% Also, please have only one work for each \bibitem.


%\end{thebibliography}
\end{document}
